% Options for packages loaded elsewhere
\PassOptionsToPackage{unicode}{hyperref}
\PassOptionsToPackage{hyphens}{url}
%
\documentclass[
]{article}
\usepackage{lmodern}
\usepackage{amssymb,amsmath}
\usepackage{ifxetex,ifluatex}
\ifnum 0\ifxetex 1\fi\ifluatex 1\fi=0 % if pdftex
  \usepackage[T1]{fontenc}
  \usepackage[utf8]{inputenc}
  \usepackage{textcomp} % provide euro and other symbols
\else % if luatex or xetex
  \usepackage{unicode-math}
  \defaultfontfeatures{Scale=MatchLowercase}
  \defaultfontfeatures[\rmfamily]{Ligatures=TeX,Scale=1}
\fi
% Use upquote if available, for straight quotes in verbatim environments
\IfFileExists{upquote.sty}{\usepackage{upquote}}{}
\IfFileExists{microtype.sty}{% use microtype if available
  \usepackage[]{microtype}
  \UseMicrotypeSet[protrusion]{basicmath} % disable protrusion for tt fonts
}{}
\makeatletter
\@ifundefined{KOMAClassName}{% if non-KOMA class
  \IfFileExists{parskip.sty}{%
    \usepackage{parskip}
  }{% else
    \setlength{\parindent}{0pt}
    \setlength{\parskip}{6pt plus 2pt minus 1pt}}
}{% if KOMA class
  \KOMAoptions{parskip=half}}
\makeatother
\usepackage{xcolor}
\IfFileExists{xurl.sty}{\usepackage{xurl}}{} % add URL line breaks if available
\IfFileExists{bookmark.sty}{\usepackage{bookmark}}{\usepackage{hyperref}}
\hypersetup{
  hidelinks,
  pdfcreator={LaTeX via pandoc}}
\urlstyle{same} % disable monospaced font for URLs
\setlength{\emergencystretch}{3em} % prevent overfull lines
\providecommand{\tightlist}{%
  \setlength{\itemsep}{0pt}\setlength{\parskip}{0pt}}
\setcounter{secnumdepth}{-\maxdimen} % remove section numbering

\author{}
\date{}

\begin{document}

\hypertarget{header-n7}{%
\subsubsection{Introduction}\label{header-n7}}

\hypertarget{header-n43}{%
\subsubsection{Abstract}\label{header-n43}}

\hypertarget{header-n44}{%
\subsubsection{Definitions}\label{header-n44}}

\hypertarget{header-n46}{%
\paragraph{Defining a Linkage}\label{header-n46}}

This project will look at structures known as planar linkages. These are
composed of a simple, connected and finite graph \(L\) paired with a
length function \(\ell: E(L) \rarr \mathbb R^+\). The notation
\((L,\ell)\) denotes this linkage.

These linkages should be intuitively thought of as a series of rigid
rods in the real plane, joined with rotating joints at the ends. This
project shall ignore the possibility of constraints on movement of this
linkage brought about by collision of these rods.

\hypertarget{header-n12}{%
\paragraph{Defining a Configuration Space}\label{header-n12}}

The way we shall relate these linkages to topological objects is through
the notion of a configuration space (sometimes called a moduli space in
other literature on the topic). The idea behind this is to look at the
unique configurations that our linkage can be manipulated into. The
notion of closeness in our configuration space comes from making small
adjustments to the angles subtended from the rods in our linkages.

The configuration space \(M\) of a linkage \((L,\ell )\) shall be
defined as follows:

\( M(L,\ell) := \{(\boldsymbol\alpha_1,..., \boldsymbol\alpha_n) \in (\mathbb S^1)^n | for \, any \, cycle \, l_i, ..., l_j\, in \, L: \sum_{k=i}^j \boldsymbol\alpha_k \ell_k = 0 \}\)
/ modulo

\hypertarget{header-n13}{%
\subparagraph{What we can hope to learn about linkages from
topology}\label{header-n13}}

\hypertarget{header-n14}{%
\subparagraph{What we can hope to learn about topology from
linkages}\label{header-n14}}

\hypertarget{header-n15}{%
\subsubsection{Finding the Configuration Space of a
Linkage}\label{header-n15}}

\hypertarget{header-n16}{%
\paragraph{Basic Idea}\label{header-n16}}

This section would introduce the configuration space with some basic
results like talking about arms and polygons and what configuration
spaces they provide.

To expand on this maybe talk about studying ranges and if that leads to
a quotient topological space and if we can work with that (needs more
research)

\hypertarget{header-n20}{%
\paragraph{Potential Topic Subtitles}\label{header-n20}}

\hypertarget{header-n21}{%
\subparagraph{Examples Finding Configuration Space of a
Linkage}\label{header-n21}}

\hypertarget{header-n22}{%
\subparagraph{Defining Homotopies between linkage states and
Manifolds}\label{header-n22}}

\hypertarget{header-n23}{%
\subparagraph{Considering Permutations of the Length function on our
Graph}\label{header-n23}}

In this section I should look at Simplex's as described in Kevin Walkers
Paper as well as discussing the different types of structures we can
form with an arm and a switch.

\hypertarget{header-n25}{%
\subparagraph{Looking at the ranges our coordinates can take and
potential linkage -\textgreater{} manifold algorithm}\label{header-n25}}

\hypertarget{header-n26}{%
\subsubsection{Finding Linkages for a Given Manifold}\label{header-n26}}

\hypertarget{header-n27}{%
\paragraph{Basic Idea}\label{header-n27}}

This is where I really want to have a result describing an algorithm
that will give you a linkage for any 2-manifold given a genus and an
orientability. However, I can't explain that, and haven't seen the
result anywhere so might be a bit of a stretch. So this topic at the
moment can talk about the degrees of freedom of a linkage, and list some
manifolds and corresponding linkages.

\hypertarget{header-n29}{%
\paragraph{Potential Topic Subtitles}\label{header-n29}}

\hypertarget{header-n30}{%
\subparagraph{Mapping Manifold Properties to
Linkages}\label{header-n30}}

\hypertarget{header-n31}{%
\subparagraph{Degrees of Freedom}\label{header-n31}}

\hypertarget{header-n32}{%
\subparagraph{The Classification Theorem for
2-Manifolds}\label{header-n32}}

\hypertarget{header-n33}{%
\subparagraph{\texorpdfstring{Linkage for the orientable 2-manifold with
genus
\(2^n\)}{Linkage for the orientable 2-manifold with genus 2\^{}n}}\label{header-n33}}

\hypertarget{header-n34}{%
\subsubsection{The Effect of Transforming Configuration
Spaces}\label{header-n34}}

\hypertarget{header-n35}{%
\paragraph{Basic Idea}\label{header-n35}}

Looking at the Siefert-van Kampen Theorem on the rectangle and types of
looping, maybe more to play with here looking at different linkages, but
following the basic idea of taking the configuration space, manipulating
it and then looking at what that gives us when we map it back to the
linkage.

\hypertarget{header-n37}{%
\paragraph{Potential Topic Titles}\label{header-n37}}

\hypertarget{header-n38}{%
\subparagraph{The fundamental group of our configuration
space}\label{header-n38}}

\hypertarget{header-n39}{%
\subparagraph{What it means to continuously deform our configuration
space}\label{header-n39}}

\hypertarget{header-n40}{%
\subparagraph{Siefert-van Kampen Theorem and the
rectangle}\label{header-n40}}

\end{document}
